\documentclass[a4paper,12pt,twoside]{article}
	\newcommand{\ig}[1]{}
	\def \be {\begin{equation}}
	\def \ee {\end{equation}}
	
	\def \dd  {{\rm d}}
	%\newcommand{\dd}{{\rm d}}
	\newcommand{\mail}[1]{{\href{mailto:#1}{#1}}}
	\newcommand{\ftplink}[1]{{\href{ftp://#1}{#1}}}
	
	\newcommand{\haut}{\text{ht}}
	\newcommand{\sgn}{\text{sign}}
	\newcommand{\ora}{\overrightarrow}
	%
	%create column vector
	\newcommand{\cvec}[3]{\begin{pmatrix}#1\\#2\\#3\end{pmatrix}}
	\newcommand{\cvect}[2]{\begin{pmatrix}#1\\#2\end{pmatrix}}
	
	\newcommand{\titleTP}[1]{\title{\vspace{-2.5cm} #1}}
	
	%joel:
	\newcommand{\req}[1]{ Eq.\,(\ref{#1})} % ref specialisé
	\newcommand{\rfi}[1]{ Fig.\,\ref{#1}}	% ref spécialisé
	\newcommand{\sib}[1]{\,[\si{#1}]}	% si entre []
	\newcommand{\refb}[1]{\,\ref{#1}}	% ?
	\newcommand{\SIe}[3]{(#1\pm\,#2)\,\si{#3}}	% SI + erreur
	\newcommand{\SIeo}[4]{(#1\pm\,#2)\cdot 10^{#3}\,\si{#4}} % SI + erreur + 10^x
	
	\newcommand{\olra}{\overleftrightarrow}
	
	\usepackage{ifthen,changepage}
	\newcommand{\ifEvenPagebreak}{
		\checkoddpage
		\ifthenelse{\boolean{oddpage}}{}{
			\hfill
			\vfill
			\thispagestyle{empty}
			\pagebreak}
	}
	

		\usepackage{graphicx} % pour l'inclusion de figures en eps,pdf,jpg
		\usepackage{amsmath} % quelques symboles mathematiques en plus
		\usepackage{amsthm} % thm definition
		\usepackage{numprint}
		\usepackage[francais]{babel}
		\usepackage[utf8]{inputenc}
		\usepackage[colorlinks,bookmarks=false,linkcolor=blue,urlcolor=blue]{hyperref}
		\usepackage[toc,page]{appendix}
		\usepackage{fancyhdr}
		%\usepackage{wasysym} % symbol supp comme l'éclaire
		\usepackage{amssymb}
		\usepackage{esint}
		\usepackage{wrapfig}
		
		%joel:
		\usepackage{siunitx} % affichage des unités SI d'une manière correcte et 			
	
	\newcommand{\thmDef}{
		\RequirePackage{amsthm}
		
		\newtheoremstyle{def}  % follow `plain` defaults but change HEADSPACE.
		{\topsep}   % ABOVESPACE
		{\topsep}   % BELOWSPACE
		{\normalfont}  % BODYFONT
		{0pt}       % INDENT (empty value is the same as 0pt)
		{\bfseries} % HEADFONT
		{.}         % HEADPUNCT
		{\newline}  % HEADSPACE. `plain` default: {5pt plus 1pt minus 1pt}
		{}          % CUSTOM-HEAD-SPEC
		
		\newtheoremstyle{thm}  % follow `plain` defaults but change HEADSPACE.
		{\topsep}   % ABOVESPACE
		{\topsep}   % BELOWSPACE
		{\itshape}  % BODYFONT
		{0pt}       % INDENT (empty value is the same as 0pt)
		{\bfseries} % HEADFONT
		{.}         % HEADPUNCT
		{\newline}  % HEADSPACE. `plain` default: {5pt plus 1pt minus 1pt}
		{}          % CUSTOM-HEAD-SPEC
		
		\newtheorem{theorem}{Théorème}
		
		\theoremstyle{def}
		\newtheorem{definition}{\underline {Définition}}[section]
		
	}
	
	\newcommand{\texteStylePhyNum}{ % texte style
		\paperheight=297mm
		\paperwidth=210mm
		
		\setlength{\textheight}{235mm}
		\setlength{\topmargin}{-1.2cm} % pour centrer la page verticalement
		%\setlength{\footskip}{5mm}
		\setlength{\textwidth}{15cm}
		\setlength{\oddsidemargin}{0.56cm}
		\setlength{\evensidemargin}{0.56cm}
	}
	
	\newcommand{\texteStyleTP}{ % texte style
		%\paperheight=297mm
		%\paperwidth=210mm
		\usepackage[left = 2cm,top = 2cm,top = 2.5cm,bottom=1.5cm]{geometry} % pour les bonnes marges
	}
	
	\newcommand{\texteStyleLM}{ % texte style
		%\paperheight=297mm
		%\paperwidth=210mm
		\usepackage[left = 2cm,top = 2cm,top = 2.5cm,bottom=1.5cm]{geometry} % pour les bonnes marges
		
	}
	
	\newcommand{\texteParagraphOption}{
		% Quelques options pour les espacements entre lignes, l'identation 
		% des nouveaux paragraphes, et l'espacement entre paragraphes
		\baselineskip=16pt
		\parindent=15pt
		\parskip=5pt
	}
	
	\newcommand{\pageHeaderFooter}[2]{
		\RequirePackage{fancyhdr}
		\pagestyle{fancy}
		\lhead{#1}
		\rhead{\today}
		\chead{ #2}
		\cfoot{\thepage}
	}

\usepackage{courier}

\usepackage{listings}
\usepackage{color}

\definecolor{dkgreen}{rgb}{0,0.6,0}
\definecolor{gray}{rgb}{0.2,0.2,0.2}
\definecolor{mauve}{rgb}{0.58,0,0.82}

\lstset{frame=l,
	language=sh,
	aboveskip=3mm,
	belowskip=3mm,
	showstringspaces=false,
	columns=flexible,
	basicstyle={\small\ttfamily},
	numbers=left,
	numberstyle=\tiny\color{gray},
	keywordstyle=\color{blue},
	commentstyle=\color{dkgreen},
	stringstyle=\color{mauve},
	breaklines=true,
	breakatwhitespace=true,
	tabsize=3
}
\usepackage{enumitem}
\texteStylePhyNum
\pageHeaderFooter{Manuelle utilisateur}{}
\begin{document}
	

		\baselineskip=16pt
		\parindent=15pt
		\parskip=5pt
		
		\title{Manuelle utilisateur pour Chicken Bot}
		\date{\today}
		\author{ChickenStorm}
		\maketitle
		
	
		\section{introduction}
		
		\section{commande}
		
		\subsection{!vote}
		la commande \verb|!vote| permet de créer et de gérer des votes.
		
		Pour voter dans un vote déjà existant, il suffit d'entrer \verb|!vote id R|, où \verb|id| est l'identifiant (un nombre) du vote (pas de panique le bot donnera cette identifiant) et \verb|R| le numéro de la réponse choisit. Si vous pensez avoir perdu l'identifiant du vote vous pouvez entrer  \verb|!vote list| (section \ref{seq:cmd:sub!vote:sub:list}) pour voir la liste des votes auxquels vous avez accès.
		
		\ig{
			\subsubsection{!vote help}
			cette commande permet d'afficher l'aide pour la commande vote
		}
		\subsubsection{!vote create}\label{seq:cmd:sub!vote:sub:create}
		
		Permet de créer un vote, la syntaxe est la suivante (en une ligne).
		\begin{lstlisting}
		!vote create Q;(role_1,role_2,role_3,...,role_N); (R_1,R_2,R_3,...,R_N) 
		\end{lstlisting}
		
		créera un vote dont la question est \verb|Q|, les réponses \verb|R_1|, \verb|R_2|, $\dots$ ,\verb|R_N|. Les personnes qui pourront voter et voir le vote sont ceux qui se trouvent dans au moins un des rôles listés; pour savoir quelle numéro correspond à quelle numéro, entrez la commande \verb|!roleList| (section \ref{seq:cmd:sub:!roleList}).
		
		Par exemple
		\begin{lstlisting}
		!vote create comment trouvez vous la fonction de vote ?;(0,5);(tres bien,geniale,fantastique) 
		\end{lstlisting}
		créera un vote comme suit :
		
		\begin{minipage}{0.01\textwidth}
			\rule{0.4pt}{4.5cm}
		\end{minipage}
		\begin{minipage}{.95\textwidth}
			
			vote créé avec succès\\
			l'id de ce vote est $1$\\
			Pour voter entrez "!vote 1 n" où n est un nombre de 1 à 3 pour le choix correspondant : \\
			vote : comment trouvez vous la fonction de vote ?
			\begin{enumerate}[label =\arabic*)]
				 \item très bien
				 \item géniale
				 \item fantastique
			\end{enumerate}
		\end{minipage}
		\subsubsection{!vote list}\label{seq:cmd:sub!vote:sub:list}
		Pour voir la liste des votes auxquels vous avez accès, entrez \verb|!vote list|.
		\subsubsection{!vote result}
		
		Pour voir le résultat d'un vote, entrez \verb|!vote result id|, où \verb|id| est l'identifiant du vote. 
		
		\subsubsection{!vote close}
		Pour fermer un vote, entrez \verb|!vote cl id|, où \verb|id| est l'identifiant du vote. Seule le propriétaire du vote peut le fermer.
		
\subsection{!roleList}\label{seq:cmd:sub:!roleList}
affiche la liste des rôles avec le nombre correspondant. ce nombre est utilisé pour la faction de création de vote (section \ref{seq:cmd:sub!vote:sub:create}).
		
\end{document}